\documentclass[a4paper,11pt]{article}
\usepackage[utf8]{inputenc}
\usepackage{minted}
\usepackage{svg}
\usepackage{float}
\usepackage{hyperref}
\usepackage[most]{tcolorbox}

\newtcolorbox{quotationbox}[1][]{colback=gray!15,colframe=black,fonttitle=\bfseries,title={#1},sharp corners}

\title{
    \textbf{Final project: n-particle simulator}
}

\author{Matvei Kazharov}
\date{Spring 2022}

\begin{document}

\maketitle

\section*{Introduction}

\begin{quotationbox}{}
    Develop an n-particle simulator that runs in linear time and create
    parallel implementations whose speedup is proportional to the number of
    processors used. You may use the provided implementations that un in polynomial time.
    Benchmark the results.
\end{quotationbox}

\section{Implementation}
I have chosen to use C++ for the project. The provided polynomial
implementations were mostly developed in C-style. As such, I looked at them for
the inspiration and formulas but wrote my own version from scratch using OOP.
For I/O, \texttt{fmt} library is used. For getting and setting options, I chose \texttt{cxxopts}.

\subsection{Serial}
The key idea is to place particles on a grid in order to easily select the
neighbors affecting the current particle. The choice of parallelization
mechanism (or lack of thereof) does not change this.

The grid is represented as a vector of forward-linked lists, each list is a cell.
The dimensions of the cell and the particle coordinates within the grid are determined by
the selected cutoff distance. When applying the forces, we take particles from
the cell above, below, left and right of the current cell the particle is in, which
leads to a linear dependency between the number of particles and the number of
computations required to simulate them.

\subsection{Pthreads}
When using pthreads, we divide the work between the specified number of threads
by giving them \texttt{start} and \texttt{stop} indices. Fine-grained locks are
used for the grid, so we only ever lock the cell we are need to work on. This
leads to reduced waiting. As such, there is a vector of locks, one for each cell.
Furthermore, in C++20 there is a new type called \texttt{std::barrier}, which
came in handy.  Two barriers are used. The first one right after applying the
forces to ensure we do not move a particle that can be used by someone else in
the computation of the previous step. The second one before outputting the
result to a file to be certain all the values were updated.  Here's the relevant
snippet:
\begin{minted}{cpp}
// we don't want to move a particle that could be used on the previous calculation
barrier.arrive_and_wait();
for (int i = start; i < stop; i++){
    auto old_cell_index = grid.getCellIndex(particles[i]);
    particles.Move(particles[i]);
    grid.CheckMove(particles[i], old_cell_index);
}
barrier.arrive_and_wait(); // to ensure all the particles are moved before a save
if (main && save && step % saveFreq == 0) {
    particles.Save(dump_file);
}
\end{minted}
\begin{figure}[H]
    \includesvg[scale=0.7]{pthreads-speedup.svg}
    \centering
    \caption{Pthreads speedup}
\end{figure}
I could not get a clear 2x speedup, which is most certainly due to the overhead
of spawning threads and other setup issues.  The spawning itself is included in
the benchmark, so ir provides a certain constant addition to the overall time.
The results seem very satisfactory nevertheless.
\subsection{OpenMP}
OpenMP implementation looks even cleaner and mimics the pthreads version, with
locks and barriers, so there is nothing much to say about it.
\begin{figure}[H]
    \includesvg[scale=0.7]{openmp-speedup.svg}
    \centering
    \caption{OpenMP speedup}
\end{figure}

As we can see, OpenMP seems a bit more lightweight for the task, achieving a
speedup of over 12 times for 16 threads. In addition to a potentially lighter
cost of thread spawning, I highly suspect that the \texttt{std::barrier} is
slower than the OPenMP barrier.

\subsection{OpenMPI}
OpenMPI is by far the hardest implementation, because it requires a fair bit of redesign.
In the end, I separated the work by rows of the grid. Each process has its own
partial grid (to not waste memory) that is aware of the offset in an imaginary
total grid. We initially broadcast the generated particles to all of the processes, then
add the relevant ones to our local grid.
After the computation of the forces and moving the particles, we asynchronously
send them to the next and previous processes (because they need the updates to
be able to compute forces on the subsequent step). We also send particles that
no longer belong to our grid (the ones that no longer belong to our cell after
updating the coordinates)
With that done, we use a blocking receive to accept the particles from the
neighboring processes. This blocking call acts as an implicit barrier.
We can think of it as two players tossing two balls: they toss their balls in
the direction of their opponent and immediately start waiting for the opponent's
ball. Thus, neither of them can proceed until both finish the I/O on this step.
Since we can't write to the same file with I/O anymore, I chose to create csv
files equal to the number of spawned processes.  The fields are the step and the
coordinates. Later on, they can be easily merged for a complete file ready for
simulation.
\begin{figure}[H]
    \includesvg[scale=0.7]{openmpi-speedup.svg}
    \centering
    \caption{OpenMPI speedup}
\end{figure}
From the benchmark we can see that the speedup is the highest of them all.
Perhaps the partial grid was a good option. But it must be noted that the
message passing would be much slower on a real distributed network of computers,
so this simulation is not entirely accurate.
\section{Overall comparison}
\begin{figure}[H]
    \includesvg[scale=0.7]{comparison.svg}
    \centering
    \caption{Parallelization comparison}
\end{figure}
From the plot of the particle number vs time (for 1000-step simulation), it seems that
OpenMPI somehow wins. The serial implementation is better on smaller sizes, as
expected, because it does not have the overhead of parallel systems.
Pthreads and OpenMP perform pretty uniformly in terms of the shape, but OpenMP
is faster.


\end{document}
